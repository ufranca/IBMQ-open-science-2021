\documentclass[%
 reprint,
%superscriptaddress,
%groupedaddress,
%unsortedaddress,
%runinaddress,
%frontmatterverbose, 
%preprint,
%showpacs,preprintnumbers,
%nofootinbib,
%nobibnotes,
%bibnotes,
 amsmath,amssymb,
 aps,
%pra,
%prb,
rmp,
%prl,
%prstab,
%prstper,
%floatfix,
]{revtex4-1}

\usepackage{graphicx}% Include figure files
\usepackage{dcolumn}% Align table columns on decimal point
\usepackage{bm}% bold math
\usepackage[dvipsnames]{xcolor}
\usepackage{amssymb}
\usepackage{amsmath}

\usepackage[urlcolor=blue, colorlinks=false, bookmarks=false]{hyperref}
%\usepackage[mathlines]{lineno}% Enable numbering of text and display math
%\linenumbers\relax % Commence numbering lines

%\usepackage[showframe,%Uncomment any one of the following lines to test 
%%scale=0.7, marginratio={1:1, 2:3}, ignoreall,% default settings
%%text={7in,10in},centering,
%%margin=1.5in,
%%total={6.5in,8.75in}, top=1.2in, left=0.9in, includefoot,
%%height=10in,a5paper,hmargin={3cm,0.8in},
%]{geometry}
% \newcommand{\AJ}[1]{\textcolor{red} {AJ: #1}}
% \newcommand{\HP}[1]{\textcolor{blue} {HP: #1}}
% \newcommand{\BJ}[1]{\textcolor{purple} {BJ: #1}}
% \newcommand{\AW}[1]{\textcolor{orange} {AW: #1}}
% \newcommand{\jay}[1]{\textcolor{green} {Jay: #1}}


\begin{document}

%\preprint{APS/123-QED}

\title{Supplemental Material for The Open Science Prize - IBM Quantum 2021}% Force line breaks with \\
%\thanks{A footnote to the article title}%

\author{Urbano L. França}
\affiliation{TBD}%Lines break automatically or can be forced with \\
\email{urbano.franca@gmail.com}

%\author{Second Author}%
% \email{Second.Author@institution.edu}
%\affiliation{%
% Authors' institution and/or address\\
% This line break forced with \textbackslash\textbackslash
%}%

\date{\today}% It is always \today, today,
             %  but any date may be explicitly specified

\begin{abstract}
% Defining the right metrics to properly represent the performance of a quantum computer is critical to both users and developers of a computing system. In this white paper, we identify three key attributes for quantum computing performance: quality, speed, and scale. Quality and scale are measured by quantum volume and number of qubits, respectively. Using an update to the quantum volume experiments, we propose a speed benchmark that allows the measurement of Circuit Layer Operations Per Second (CLOPS) and identify how both classical and quantum components play a role in improving performance. We prescribe a procedure for measuring CLOPS and use it to characterize the performance of some IBM Quantum systems. 
\end{abstract}

\maketitle
%\tableofcontents
\section{\label{sec:intro}Introduction}

bla bla blabla bla blabla bla blabla bla blabla bla blabla bla blabla bla blabla bla blabla bla bla
bla bla blabla bla blabla bla blabla bla blabla bla bla

bla bla blabla bla blabla bla blabla bla bla

bla bla blabla bla blabla bla bla

\cite{Alexander20}

\cite{Bravyi21}

\cite{Corcoles21a}

\cite{Earnest21}

\cite{Heras14}

\cite{Kandala19}

\cite{Krantz19}

\cite{Lloyd96}

\cite{Nation21}

\cite{Salathe15}

\cite{Satoh21}

\cite{Smith19}

\cite{Tacchino19}

\cite{Temme17}

\cite{Kandala19}

\cite{Kim21}

\cite{Gokhale20}

\section{\label{sec:conclusion}Summary}

Performance benchmarks have always been difficult to properly engineer for classical computer systems, and quantum systems add both result quality and interaction with classical systems into the equation. We have shown that low level, single dimension benchmarks do not properly express the performance that user's see from the system. Instead it is necessary to create holistic benchmarks that capture all of the components that will translate to performance on real world applications but not be overly cumbersome to execute. We have defined a CLOPS benchmark that captures many of the necessary aspects for running user applications with good performance, and provided examples of using the benchmark to find current bottlenecks in the system.

\bibliography{references}



\end{document}
%